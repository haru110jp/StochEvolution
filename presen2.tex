\documentclass[dvipdfmx,fleqn]{beamer}
%\documentclass[dvipdfmx,fleqn,handout]{beamer}
\usepackage{amsmath,amssymb,amsthm,pxjahyper}
\usepackage{minijs}
\usepackage{otf}
\mode<presentation>
{
  \usetheme{Hannover}
}
\usecolortheme{dove}
\usefonttheme{professionalfonts}
\setbeamertemplate{frametitle}[default][center]
\setbeamertemplate{navigation symbols}{}
\setbeamercovered{transparent}
\setbeamertemplate{footline}[page number]
\setbeamerfont{footline}{size=\scriptsize}

\setcounter{framenumber}{0}

\title{確率進化モデル}
\author{山岸 敦}
\institute{東京大学経済学部3年・尾山ゼミ}
\date{2014 11/10}

\begin{document}
\frame{\titlepage}

\section{} %ヨコに目次を表示するためにつけてます(これ消すと、他のも消える)

\section{現在の進行状況}
\begin{frame}\frametitle{現在の進行状況}
\begin{itemize}\setlength{\parskip}{0.5em}
\item
Ellisonモデルからの拡張となるネットワークの話は、いったん保留
\item
ネットワーク班の進捗を待ちつつ、別の確率進化モデルのプログラムを書くことに
\item
Stochastic Fictitious PlayとLog-Linear Dynamicsに沿って行動するプレイヤーのクラスを作成
\end{itemize}
\end{frame}



\section{Stochastic Fictitious Play}
\begin{frame}
\frametitle{Fictitious Playのおさらい}
\begin{itemize}\setlength{\parskip}{0.5em}
\item
"Stochastic" Fictitious Playは、夏学期にやったFictitious Playの応用版です。
\item
念のため、Fictitious Playを復習しましょう……
\item
覚えておいて欲しいのは、{\bf 新たな情報に対するウェートがどんどん低下している}ということです
\item
これは、相手の真の(最初からずっと不変の)真の混合戦略を学習しようとしている、と解釈できます
\end{itemize}
\end{frame}

\begin{frame}
\frametitle{Stochastic Fictitious Play}
\begin{itemize}\setlength{\parskip}{0.5em}
\item
Stochastic Fictitious Playでは、ウェートが低下しません。Fictitious Playより現在を重視しています
\item
「相手の真の戦略分布が途中で変わるかもしれない」と想定していることを示唆
\item
"プレーヤ2の、プレイヤー1が関してが戦略iをとってくる確率の予想”を$\theta_{1it}$と表し、$\epsilon$は定数とすれば
\[
\theta_{1it+1} = \theta_{1it} + \epsilon(1_{(Player\:1\:plays\:i\,)} - \theta_{1it})
\]
\item
ただし、$1_{(Player\:1\:plays\:i\:)}$は、プレイヤー1がiをすると1で、そうでないと0です(Indicator Function)

\end{itemize}
\end{frame}

\begin{frame}
\frametitle{Stochastic Fictitious Play}
\begin{itemize}\setlength{\parskip}{0.5em}
\item
これだけでは、"Stochastic"な要素がありません
\item
利得にランダム性が導入されます。普通に期待利得を計算した後確率変数が「ボーナスorペナルティ」として加算され、それを見てから意思決定がされます
\item
$\theta_{2t} = (\theta_{21t},\theta_{22t},...)$とし、さらに$\alpha_{i1}$で相手が戦略1を取った時、戦略iのもたらす利得を表記して$\alpha_i = (\alpha_{1i},\alpha_{1i}...)$とします。$ \mathrm{e}^1_{it} $は、プレイヤー1の戦略iに入る"ボーナスポイント"です
\item
t期に、プレイヤー1が戦略iを取るとき、
\[
\theta_{2t} \cdot \alpha_i +  \mathrm{e}^1_{it} \: \ge \: \max_{j \neq j} \{\theta_{2t} \cdot \alpha_j + \mathrm{e}^1_{jt} \}
\]


\end{itemize}
\end{frame}

\begin{frame}
\frametitle{Stochastic Fictitious Play}
\begin{itemize}\setlength{\parskip}{0.5em}
\item
とりあえず動くようにはなっているので、動かしてみましょう
\item
ビジュアル面は随時改善します
\end{itemize}
\end{frame}

\section{Log-linear Dynamics}
\begin{frame}
\frametitle{Log-Linear Dynamics}
\begin{itemize}\setlength{\parskip}{0.5em}
\item
つづいて、Log-Linear Dynamicsの方に移ります
\item
未完成ですが、骨格はできています。
\item
プレイヤーのクラスができたので、あとはゲーム全体の流れを書くプログラムのなかにはめ込むだけ
\end{itemize}
\end{frame}

\begin{frame}
\frametitle{Log-Linear Dynamics}
\begin{itemize}\setlength{\parskip}{0.5em}
\item
一言で言えば、「AとB2つの戦略を取る確率の対数の差」が「利得の差」に比例するような混合戦略をとります。
\item
式を整理することで、一意的に各戦略がとられる確率を導出できます。
\item
つまり、(見た目は複雑でも)構造自体は複雑ではありません。
\item
元論文のPDFを見てみましょう
\end{itemize}
\end{frame}

\section{補足:ネットワークはどうした?}
\begin{frame}
\frametitle{ネットワークの方の進捗}
\begin{itemize}\setlength{\parskip}{0.5em}
\item
どうも基礎知識が足りない感じがしたので、合宿の後くらいからおべんきょうを開始
\item とりあえずやってみたこと
  \begin{itemize}
    \item
    coursera(無料のオンライン大学)で、M. Jacksonの「Social and Economic Networks」を受講(きのう単位取得)
    \item
    グラフ理論をかじる(どうも本の選定を間違えたようで、やる気を削がれ頓挫中…)
    \item
    ネットワークが絡む経済学の論文をいくつか読んだ(MorrisのContagionは現在進行形)
    \end{itemize}
\item
読んだ論文の中で面白かったのは、"Rumors and Social Networks" (2014)\footnote{Francis Bloch, Gabrielle Demange, Rachel Kranton. "Rumors and Social Networks." PSE Working Papers n2014-15 2014} というワーキングペーパー。

\end{itemize}
\end{frame}

\section{これからやること}
\begin{frame}
\frametitle{これからやること}
\begin{itemize}\setlength{\parskip}{0.5em}
\item
"Monotone Potential Maximizer"のところに乗ってた論文から利得表をとってきて、今までの4つのモデル(KMR, Ellison, Stochastic Fictitious Play, Log-Linear)に入れて比較する
\item
実践の中で、比較しやすいよう各プログラムを改善していく
\item
ネットワークの方も継続(卒論にできるといいなあ、と思い始めているところ)
\item
ネットワーク班で、もしやることがあれば手伝います
\end{itemize}
\end{frame}




\end{document}